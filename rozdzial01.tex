
\chapter{Cel projektu}

Celem niniejszej pracy jest projekt i~implementacja modułu zarządzania zadaniami dla systemu Gerbil (\url{http://gerbilvis.org/}). Jest to system do analizy i~wizualizacji danych wielospektralnych, rozwijany na uniwersytecie Friedrich-Alexander-Universit\"{a}t Erlangen-N\"{u}rnberg w~Laboratorium Rozpoznawania Wzorców, pod kierownictwem doktora Johannesa Jordana. Gerbil posiada zestaw wielu algorytmów przetwarzania obrazów oraz uczenia maszynowego, które przekładają się na szerokie spektrum funkcjonalności. Jednak jego słabym punktem jest warstwa zarządzania danymi oraz potok przetwarzania danych. To z~kolei powoduje niestabilność całej aplikacji. W~ramach pracy dyplomowej został zaproponowany system, który rozwiązuje wyżej wspomniane problemy. System ten pozwala na bezpieczny dostęp do danych w~całej aplikacji oraz gwarantuje zachowanie właściwego potoku przetwarzania danych. Współpraca z projektem Gerbil odbywa się w~ramach projektu \mbox{ESA Summer of Code in Space 2016} (\url{http://sophia.estec.esa.int/socis/}).
Kod źródłowy systemu można znaleźć pod adresem \url{https://github.com/ajaskier/gerbil/tree/distalpha}.
