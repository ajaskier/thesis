\chapter{Implementacja}

\section{Kompilacja projektu}
Kod źródłowy projektu można pobrać programem \lstinline$wget$ dostępnym w systemie GNU/Linux: \lstinline$wget https://github.com/ajaskier/gerbil/archive/distalpha.zip$. W~wyniku wykonania tej komendy pobrane zostanie archiwum o~nazwie \lstinline$gerbil-distalpha.zip$ zawierające kod źródłowy. Archiwum można rozpakować programem \lstinline$unzip$, który również jest dostępny w systemie \mbox{GNU/Linux:} \lstinline$unzip gerbil-distalpha.zip$. Po wykonaniu tej komendy kod źródłowy projektu znajdować się będzie w~katalogu \lstinline$gebril-distalpha$. Do zarządzania procesem kompilacji tego projektu wykorzystywane jest narzędzie \lstinline$CMake$~\cite{cmake}. Do rozwiązania zależności projektu można skorzystać z~interfejsu tego narzędzia o~nazwie \lstinline$ccmake$. W~programie \lstinline$ccmake$ należy wprowadzić ścieżki do wymaganych zależności projektu, zatwierdzić konfigurację klawiszem \lstinline$c$ po czym wygenerować szkielet projektu klawiszem \lstinline$g$. Po wykonaniu tego kroku można dokonać kompilacji projektu za pomocą programu \lstinline$make$.

\section{Integracja z~projektem Gerbil}
W skład pierwszej fazy integracji nowego systemu z~projektem Gerbil wchodzą 2 etapy:
\begin{enumerate}[labelwidth=\widthof{\ref{last-item2}},label=\arabic*.]
	\item zapewnienie funkcjonalności związanych z~reprezentacją obrazów,
	\item zapewnienie funkcjonalności związanych z~histogramami spektralnymi.
\end{enumerate}

W skład etapu pierwszego wchodzi: 
\begin{itemize}
		\item adaptacja istniejących klas zadań odpowiedzialnych za obliczenia konkretnych reprezentacji obrazów wielospektralnych do nowego interfejsu klasy \lstinline$Task$,
		\item adaptacja klasy \lstinline$ImageModel$ do nowego interfejsu klasy \lstinline$Model$,
		\item adaptacja widoków wyświetlających reprezentacje obrazów do nowego mechanizmu dostępu do danych współdzielonych.
\end{itemize}

Na drugi etap składa się:
\begin{itemize}
		\item zdefiniowanie procesu wykonania dla struktury reprezentującej histogram spektralny,
		\item adaptacja istniejących klas zadań odpowiedzialnych za obliczenie histogramu spektralnego,
		\item adaptacja klasy \lstinline$DistViewModel$ do nowego interfejsu klasy \lstinline$Model$,
		\item adaptacja widoków prezentujących histogramy spektralne do nowego mechanizmu dostępu do danych współdzielonych,	
\end{itemize}

Efektem tej fazy integracji powinna być aplikacja będąca podzbiorem funkcjonalności oryginalnej aplikacji Gerbil. Jej implementacja powinna być wystarczającym źródłem przykładów, aby zintegrować resztę komponentów projektu z~nowym systemem zarządzania danymi oraz procesem wykonania.

Etap pierwszy integracji został zakończony, aktualnie trwają prace nad ukończeniem etapu drugiego.

\section{Dalsze kierunki rozwoju systemu}
Do dalszych kierunków rozwoju należy:
\begin{itemize}
	\item dalsza integracja z~projektem Gerbil,
	\item zapewnienie możliwości anulowania zadania które zostało uruchomione.
\end{itemize}
