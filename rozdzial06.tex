\chapter{Podsumowanie}

\section{Integracja z~projektem Gerbil}
W skład pierwszej fazy integracji nowego systemu z~projektem Gerbil wchodzą 2 etapy:
\begin{enumerate}[labelwidth=\widthof{\ref{last-item2}},label=\arabic*.]
	\item zapewnienie funkcjonalności związanych z~reprezentacją obrazów,
	\item zapewnienie funkcjonalności związanych z~histogramami spektralnymi.
\end{enumerate}

W skład etapu pierwszego wchodzi: 
\begin{itemize}
		\item adaptacja istniejących klas zadań odpowiedzialnych za obliczenia konkretnych reprezentacji obrazów wielospektralnych do nowego interfejsu klasy \lstinline$Task$,
		\item adaptacja klasy \lstinline$ImageModel$ do nowego interfejsu klasy \lstinline$Model$,
		\item adaptacja widoków wyświetlających reprezentacje obrazów do nowego mechanizmu dostępu do danych współdzielonych.
\end{itemize}

Na drugi etap składa się:
\begin{itemize}
		\item zdefiniowanie procesu wykonania dla struktury reprezentującej histogram spektralny,
		\item adaptacja istniejących klas zadań odpowiedzialnych za obliczenie histogramu spektralnego,
		\item adaptacja klasy \lstinline$DistViewModel$ do nowego interfejsu klasy \lstinline$Model$,
		\item adaptacja widoków prezentujących histogramy spektralne do nowego mechanizmu dostępu do danych współdzielonych,	
\end{itemize}

Efektem tej fazy integracji powinna być aplikacja będąca podzbiorem funkcjonalności oryginalnej aplikacji Gerbil. Jej implementacja powinna być wystarczającym źródłem przykładów, aby zintegrować resztę komponentów projektu z~nowym systemem zarządzania danymi oraz procesem wykonania.

Etap pierwszy integracji został zakończony, aktualnie trwają prace nad ukończeniem etapu drugiego.

\section{Porównanie systemów}
Przewagę nowego systemu zarządzania danymi oraz procesem wykonania można wyróżnić na 2 płaszczyznach -- dostępu do danych współdzielonych oraz zapewnienia propagacji sygnałów aktualizacji danych.

\subsection{Dostęp do danych współdzielonych}

Bezpieczeństwo dostępu do danych zostało poprawione w~nowej wersji systemu.


\begin{minipage}{\textwidth}
	\begin{lstlisting}[label=dataaccess:old, caption={Przykład dostępu do danych według bieżącego systemu},alsoletter={()[].=}]
SharedDataLock ctxlock(ctx->mutex);
limiters.assign((*ctx)->dimensionality, std::make_pair(0, (*ctx)->nbins-1));
	\end{lstlisting}
\end{minipage}

Na listingu \ref{dataaccess:old} przedstawiony został sposób dostępu do danych współdzielonych w~obecnej architekturze. Z~analizy listingu wynika, że przed faktycznym dostępem do danych wykonywane jest zajęcie muteksu skojarzonego z~tą daną. Pozostawienie odpowiedzialności synchronizacji dostępu programiście, który chce ich użyć jest zabiegiem niebezpiecznym. Programista może kwestię synchronizacji pominąć, lub o~niej zapomnieć. Skutkować to może naruszeniem ochrony pamięci i~zakończeniem działania programu.

\begin{minipage}{\textwidth}
	\begin{lstlisting}[label=dataaccess:new, caption={Przykład dostępu do danych według nowego systemu},alsoletter={()[].=}]
Subscription::Lock<multi_img> lock(*sub);
multi_img* img = lock();
QPixmap pix = QPixmap::fromImage(img->export_qt(1));
	\end{lstlisting}
\end{minipage}

Na listingu \ref{dataaccess:new} został przedstawiony sposób dostępu do danych współdzielonych za pomocą nowego systemu. Z analizy listingu można wywnioskować, że w~tym systemie programista, który chce użyć danych nie jest zobowiązany do wykonywania synchronizacji dostępu. Rzecz ta należy do funkcjonalności modelu danych współdzielonych i~wykonywana jest automatycznie podczas żądania dostępu.

\subsection{Propagacja sygnałów aktualizacji danych}

Sposób definiowania procesu wykonania danych uległ diametralnej zmianie. Obecny system zarządzania procesem wykonania jest zdecentralizowany. Aby utworzenie danej~A było zlecone na skutek obliczenia danej~B, model~B musi emitować sygnał informujący o~obliczeniu danej~B, model~A posiadać slot w~którym zdefiniowana jest reakcja na taki sygnał oraz kontroler (jako element warstwy nadrzędnej) musi ustanowić połączenie pomiędzy danym sygnałem i~slotem. W~wyniku tego aby zapewnić odpowiedni proces wykonania w~systemie, należy ręcznie ustanowić bardzo wiele połączeń, co zaciemnia kod. Przykład takiego zaciemnienia może stanowić listing \ref{propagation:old}.

\begin{minipage}{\textwidth}
	\begin{lstlisting}[label=propagation:old, caption={Przykład definiowania procesu wykonania według starego systemu},alsoletter={()[].=}]
connect(lm, SIGNAL(newLabeling(const cv::Mat1s&, const QVector<QColor>&, bool)), dvc, SLOT(updateLabels(cv::Mat1s,QVector<QColor>,bool)));
connect(lm, SIGNAL(partialLabelUpdate(const cv::Mat1s&,const cv::Mat1b&)), dvc, SLOT(updateLabelsPartially(const cv::Mat1s&,const cv::Mat1b&)));
connect(dvc, SIGNAL(alterLabelRequested(short,cv::Mat1b,bool)), lm, SLOT(alterLabel(short,cv::Mat1b,bool)));
connect(illumm, SIGNAL(newIlluminantCurve(QVector<multi_img::Value>)), dvc, SIGNAL(newIlluminantCurve(QVector<multi_img::Value>)));
connect(illumm, SIGNAL(newIlluminantApplied(QVector<multi_img::Value>)), dvc, SIGNAL(newIlluminantApplied(QVector<multi_img::Value>)));
	\end{lstlisting}
\end{minipage}

W nowym systemie definiowana procesu wykonania problem ten nie istnieje. Proces wykonania jest definiowany przez deklaracje danych współdzielonych i~ich zależności, natomiast klasa \lstinline$SubscriptionManager$ zapewnia funkcjonalność informowania modeli o~potrzebie utworzenia zadania przetwarzającego dane. W~związku z~tym w~nowej architekturze systemu nie istnieje potrzeba aby ręcznie zarządzać procesem przetworzenia danych.

\section{Dalsze kierunki rozwoju systemu}
Do dalszych kierunków rozwoju należy:
\begin{itemize}
	\item dalsza integracja z~projektem Gerbil,
	\item zapewnienie możliwości anulowania zadania które zostało uruchomione.
\end{itemize}
