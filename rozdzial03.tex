\chapter{Technologie wykorzystane w~systemie Gerbil}

\index{C++} \section{C++}
System Gerbil jest rozwijany w języku C++. 

\index{STL} \subsection{STL}

\index{Qt} \section{Qt}

Qt jest platformą deweloperską wyposażoną w narzędzia pozwalające usprawnić proces wytwarzania oprogramowania oraz interfejsów użytkownika dla aplikacji desktopowych, wbudowanych bądź mobilnych.~\footnote{Qt 5.7 \url{http://doc.qt.io/qt-5/index.html}} 

Platforma Qt posiada szerokie spektrum funkcjonalności. Między innymi są to:
\begin{itemize}
	\item system meta-obiektów,
	\item mechanizm sygnałów i slotów służący do komunikacji pomiędzy obiektami,
	\item wbudowany system przynależności obiektów,
	\item wieloplatformowe wsparcie modułu wielowątkowości.
		
\end{itemize}

W systemie Gerbil jest wykorzystywane Qt w wersji 5.7.

\subsection{Sygnały i sloty}

Spośród rozrzerzeń języka C++, jakie oferuje Qt na szczególną uwagę zasługuje mechanizm sygnałów i slotów. Dzięki niemu możliwe jest skomunikowanie dwóch dowolnych obiektów w sposób alternatywny do użycia wywołań zwrotnych. 

Sygnał jest wysyłany, gdy nastąpi jakieś zdarzenie (np. naciśnięcie przycisku przez użytkownika), natomiast slot jest odpowiedzią na ten sygnał. Sygnatury sygnału i slotu muszą być zgodne. Mechanizm ten jest luźno powiązany (ang. loosely coupled). Oznacza to, że klasa emitująca sygnał nie musi być swiadoma klasy odbierającej. Sygnały i sloty pozwalają na przekazanie dowolnej liczby argumentów dowolnego typu. Sygnały muszą zostać zadeklarowane po słowie kluczowym \lstinline$signals$. Z jednym slotem można połączyć dowolną ilość sygnałów, i odwrotnie -- z jednym sygnałem można skojarzyć dowolną ilość slotów.

Wszystkie klasy korzystające z tego mechanizmu muszą w swojej deklaracji zawierać makro \lstinline$Q_OBJECT$ oraz dziedziczyć (bezpośrednio bądź pośrednio) po klasie \lstinline$QObject$.~\footnote{Signals \& Slots | Qt Core 5.7 \url{http://doc.qt.io/qt-5/signalsandslots.html}}


\subsubsection{Składnia} 
Sposób tworzenia połączeń zostanie zilustrowany na przykładzie. Za punkt wyjścia posłużą dwie klasy: \lstinline$Sender$ (Listing \ref{sender}) oraz \lstinline$Receiver$ (Listing \ref{receiver}).

\begin{minipage}{\textwidth}
	\begin{lstlisting}[label=sender,caption=Klasa Sender]
	class Sender : public QObject
	{
		Q_OBJECT
	public:
		explicit Sender(QObject *parent = 0) : QObject(parent) {}
		
	signals:
		void sendMessage(QString msg);
		
	};
	\end{lstlisting}
\end{minipage}

\begin{minipage}{\textwidth}
	\begin{lstlisting}[label=receiver, caption=Klasa Receiver]
	class Receiver : public QObject
	{
		Q_OBJECT
	public:
		explicit Receiver(QObject *parent = 0) : QObject(parent) {}
		
		void receiveMessageMethod(QString msg) {
			std::cout << "got message in method: " << msg;
		}
		
	public slots:
		void receiveMessageSlot(QString msg) {
			std::cout << "got message: " << msg;
		}
	};
	\end{lstlisting}
\end{minipage}

Z analizy listingów \ref{sender} oraz \ref{receiver} wynika, że klasa \lstinline{Sender} zawiera sygnał \lstinline{sendMessage}, natomiast klasa \lstinline{Receiver} zawiera publiczną metodę \lstinline{receiveMessageMethod} oraz publiczny slot \lstinline{receiveMessageSlot}.

W Qt występują dwa rodzaje składni pozwalające na ustanowienie połączenia. Jedna z nich (starsza) pozwala na ustanowienie połączenia jedynie pomiędzy sygnałem a sygnałem, bądź sygnałem a slotem. Drugi rodzaj składni, wprowadzony w Qt5 pozwala dodatkowo na nawiązanie połączenia pomiędzy sygnałem a metodą klasy. Na listingu \ref{connectionsyntax} przedstawiony jest zarówno stary jak i nowy zapis.

%\begin{minipage}{\textwidth}

	\begin{lstlisting}[label=connectionsyntax, caption={Składnia tworzenia połączeń między obiektami},alsoletter={()[].=}]
  Sender sender;
  Receiver receiver;
  
  //stara skladnia
  //poprawne
  QObject::connect(&sender, SIGNAL(sendMessage(QString)), &receiver, SLOT(receiveMessageSlot(QString)));
  //niepoprawne
  QObject::connect(&sender, SIGNAL(sendMessage(QString)), &receiver, SLOT(receiveMessageMethod(QString)));
  
  //nowa skladnia
  QObject::connect(&sender, &Sender::sendMessage, &receiver, &Receiver::receiveMessageMethod);
  QObject::connect(&sender, &Sender::sendMessage, &receiver, &Receiver::receiveMessageSlot);
	\end{lstlisting}
%\end{minipage}


\subsection{Wątek GUI oraz wątki robocze}

\index{boost} \section{boost}

\index{TBB} \section{TBB}

\index{OpenCV} \section{OpenCV}

