\chapter{Technologie wykorzystane w~systemie Gerbil}

Projekt jest rozwijany pod systemem operacyjnym Linux, dystrybucją Ubuntu 16.04.

\index{C++} \section{C++}
System Gerbil jest rozwijany w języku C++. Jest to język programowania ogólnego przeznaczenia, ze szczególnym zastosowaniem w tworzeniu systemów. C++ to język: 
\begin{itemize}
	\item wieloparadygmatowy -- pozwala na programowanie proceduralne, obiektowe, funkcyjne oraz ogólne,
	\item statycznie typowany -- zgodność typów jest sprawdzana w trakcie kompilacji,
	\item pozwalający na bezpośrednie zarządzanie pamięcią,
	\item tworzony według zasady zerowego narzutu - elementy tego języka oraz proste abstrakcje muszą być optymalne (nie marnować bajtów pamięci ani cyklów procesora),
	\item umożliwiający tworzenie lekkich i wydajnych abstrakcji \cite{Stroustrup}\cite{cplusplus}.
\end{itemize}

Język o takiej charakterystyce jest dobrym wyborem do implementacji systemu analizy i wizualizacji skomplikowanych danych.

W projekcie używany jest standard ISO/IEC 14882:2011 języka C++ oraz kompilator GCC w wersji 5.4.0 \cite{C++11}.

\index{STL} \subsection{STL}
STL (ang. Standard Template Library) jest biblioteką standardową języka C++. Oferuje ona szereg kontenerów, klas, obiektów funkcyjnych oraz algorytmów. Składniki te opisane są w standardzie ISO języka C++, oraz gwarantują identyczne zachowanie w każdej implementacji \cite{Stroustrup}. Ułatwia to tworzenie aplikacji wieloplatformowych.Dzięki gotowym rozwiązaniom zawartym w bibliotece standardowej, proces wytwarzania oprogramowania zyskuje na prostocie i efektywności. 

\index{shared\_ptr} \subsubsection{\lstinline$shared_ptr$}
\lstinline$shared_ptr$ jest typem umożliwiającym reprezentację własności wspólnej. Wykorzystywany jest w sytuacjach, gdy dwa (lub więcej) fragmenty kodu wymagają dostępu do danych, podczas gdy żaden nie jest odpowiedzialny za usunięcie tych danych. Obiekt \lstinline$shared_ptr$ jest rodzajem wskaźnika z licznikiem wystąpień. Jeśli liczba obiektów wskazujących na konkretną daną spadnie do zera, dana ta jest usuwana \cite{Stroustrup}. 

\index{mutex} \subsubsection{\lstinline$mutex$}
Muteks jest obiektem typu \lstinline$mutex$, służącym do reprezentowania wyłącznych praw dostępu do konkretnego zasobu. Wykorzystuje się go do ochrony przed wyścigami do danych oraz synchronizacji dostępu do danych współdzielonych między wątkami.

Muteks może być w posiadaniu tylko jednego wątku na raz. Zajęcie muteksu jest równoznaczne z nabyciem wyłącznych praw własności do niego. Operacja zajmowania muteksu jest blokująca. Zwolnienie muteksu oznacza zrzeczenie się z prawa własności do niego. Daje to możliwość zajęcia muteksu przez inne oczekujące wątki \cite{Stroustrup}.

\subsubsection{\lstinline$condition_variable$}

\index{Qt} \section{Qt}

Qt jest platformą programistyczną wyposażoną w narzędzia pozwalające usprawnić proces wytwarzania oprogramowania oraz interfejsów użytkownika dla aplikacji desktopowych, wbudowanych bądź mobilnych \cite{Qtdoc}.

Platforma Qt posiada szerokie spektrum funkcjonalności. Między innymi są to:
\begin{itemize}
	\item system meta-obiektów,
	\item mechanizm sygnałów i slotów służący do komunikacji pomiędzy obiektami,
	\item wbudowany system przynależności obiektów,
	\item wieloplatformowe wsparcie modułu wielowątkowości.
		
\end{itemize}

W systemie Gerbil jest wykorzystywane Qt w wersji 5.7.

\index{sygnał} \index{slot} \subsection{Sygnały i sloty}

Spośród rozszerzeń języka C++, jakie oferuje Qt, na szczególną uwagę zasługuje mechanizm sygnałów i slotów. Dzięki niemu możliwe jest skomunikowanie dwóch dowolnych obiektów w sposób alternatywny do użycia wywołań zwrotnych. 

Sygnał jest wysyłany, gdy nastąpi jakieś zdarzenie (np. naciśnięcie przycisku przez użytkownika), natomiast slot jest odpowiedzią na ten sygnał. Sygnatury sygnału i slotu muszą być zgodne. Mechanizm ten jest luźno powiązany (ang. loosely coupled). Oznacza to, że klasa emitująca sygnał nie musi być świadoma klasy odbierającej. Sygnały i sloty pozwalają na przekazanie dowolnej liczby argumentów dowolnego typu. Sygnały muszą zostać zadeklarowane po słowie kluczowym \lstinline$signals$. Z jednym slotem można połączyć dowolną ilość sygnałów, i odwrotnie -- z jednym sygnałem można skojarzyć dowolną ilość slotów.

Wszystkie klasy korzystające z tego mechanizmu muszą w swojej deklaracji zawierać makro \lstinline$Q_OBJECT$ oraz dziedziczyć (bezpośrednio bądź pośrednio) po klasie \lstinline$QObject$ \cite{Qtdoc}.


\subsubsection{Składnia} 
Sposób tworzenia połączeń zostanie zilustrowany na przykładzie. Za punkt wyjścia posłużą dwie klasy: \lstinline$Sender$ (Listing \ref{sender}) oraz \lstinline$Receiver$ (Listing \ref{receiver}).

\begin{minipage}{\textwidth}
	\begin{lstlisting}[label=sender,caption=Klasa Sender]
class Sender : public QObject
{
	Q_OBJECT
public:
	explicit Sender(QObject *parent = 0) : QObject(parent) {}
	
signals:
	void sendMessage(QString msg);
	
};
	\end{lstlisting}
\end{minipage}

\begin{minipage}{\textwidth}
	\begin{lstlisting}[label=receiver, caption=Klasa Receiver]
class Receiver : public QObject
{
	Q_OBJECT
public:
	explicit Receiver(QObject *parent = 0) : QObject(parent) {}
	
	void receiveMessageMethod(QString msg) {
		std::cout << "got message in method: " << msg;
	}
	
public slots:
	void receiveMessageSlot(QString msg) {
		std::cout << "got message: " << msg;
	}
};
	\end{lstlisting}
\end{minipage}

Z analizy listingów \ref{sender} oraz \ref{receiver} wynika, że klasa \lstinline{Sender} zawiera sygnał \lstinline{sendMessage}, natomiast klasa \lstinline{Receiver} zawiera publiczną metodę \lstinline{receiveMessageMethod} oraz publiczny slot \lstinline{receiveMessageSlot}.

W Qt występują dwa rodzaje składni pozwalające na ustanowienie połączenia. Jedna z nich (starsza) pozwala na ustanowienie połączenia jedynie pomiędzy sygnałem a sygnałem, bądź sygnałem a slotem. Drugi rodzaj składni, wprowadzony w Qt5, pozwala dodatkowo na nawiązanie połączenia pomiędzy sygnałem a metodą klasy. Na listingu \ref{connectionsyntax} przedstawiony jest zarówno stary jak i nowy zapis.

\begin{minipage}{\textwidth}
	\begin{lstlisting}[label=connectionsyntax, caption={Składnia tworzenia połączeń między obiektami},alsoletter={()[].=}]
Sender sender;
Receiver receiver;

//stara skladnia
//poprawne
QObject::connect(&sender, SIGNAL(sendMessage(QString)), &receiver, SLOT(receiveMessageSlot(QString)));
//niepoprawne
QObject::connect(&sender, SIGNAL(sendMessage(QString)), &receiver, SLOT(receiveMessageMethod(QString)));

//nowa skladnia
QObject::connect(&sender, &Sender::sendMessage, &receiver, &Receiver::receiveMessageMethod);
QObject::connect(&sender, &Sender::sendMessage, &receiver, &Receiver::receiveMessageSlot);
	\end{lstlisting}
\end{minipage}

Mechanizm sygnałów i slotów jest powszechnie wykorzystywany w systemie Gerbil do ustanowienia komunikacji pomiędzy obiektami. Używana jest zarówno stara jak i nowa składnia.

\subsection{Wątek GUI oraz wątki robocze}
GUI (ang. Graphical User Interface) jest graficznym interfejsem użytkownika. Każda aplikacja jest uruchamiana w osobnym wątku. Jest on nazywany wątkiem głównym (bądź "wątkiem GUI" w aplikacjach Qt). Interfejs użytkownika rozwijany w Qt musi zostać uruchomiony w tym wątku. Wszystkie komponenty graficznego interfejsu użytkownika (ang. widget) oraz kilka klas pochodnych nie zadziałają w wątkach pobocznych. Wątki poboczne są często nazywane "wątkami roboczymi", ponieważ wykorzystywane są aby odciążyć główny wątek od skomplikowanych obliczeń \cite{Qtdoc}. Gdyby te obliczenia były wykonywane w głównym wątku, aplikacja przestałaby być responsywna podczas obliczeń, co jest efektem niepożądanym.

\index{Boost} \section{Boost}
Boost jest kolekcją bibliotek do języka C++. Biblioteki te poszerzają funkcjonalności tego języka \cite{boost}. Wiele z bibliotek rozwijanych przez Boost zostało włączonych do standardu C++. Z perspektywy systemu Gerbil na specjalną uwagę zasługuje Boost.Any.



\subsection{Boost.Any}
W języku C++ kwestia przechowania obiektów dowolnego typu jest problematyczna, ponieważ jest to język statycznie typowany. 

\subsubsection{\lstinline$void*$}
W czystym C++ można użyć \lstinline$void*$. Do zmiennej typu \lstinline$void*$ można przypisać wskaźnik dowolnego typu poza wskaźnikiem do funkcji oraz wskaźnikiem do składowej. Aby użyć takiej zmiennej należy dokonać jawnej konwersji \lstinline$static_cast$. Do zastosowania \lstinline$void*$ w kodzie wysokopoziomowym należy podchodzić z rezerwą. Może to wskazywać na błędy projektowe \cite{Stroustrup}.

\subsubsection{\lstinline$boost::any$}
Rozwiązaniem, które z powodzeniem można stosować w kodzie wysokopoziomowym jest właśnie klasa \lstinline$boost::any$. Jego zdecydowaną przewagą nad \lstinline$void*$ jest bezpieczny typowo interfejs. Jest to kontener opakowujący pojedynczy obiekt niemal dowolnego typu (obiekt musi posiadać możliwość inicjalizacji na bazie innego obiektu tego typu -- ang. copy-constructible).
Aby użyć obiektu przechowywanego przez \lstinline$boost::any$ należy dokonać rzutowania \lstinline$boost::any_cast$. Jeżeli zostanie podany typ, na który obiekt nie może zostać zrzutowany, zostanie zgłoszony wyjątek \lstinline$boost::bad_any_cast$ \cite{boost}.

\index{TBB} \section{Intel Threading Building Blocks}
Intel Threading Building Blocks (TBB) jest biblioteką szablonów ułatwiającą programowanie równoległe. W swojej ofercie posiada gotowe struktury danych oraz zrównoleglone algorytmy~\cite{tbb}.

W systemie Gerbil biblioteka TBB używana jest głownie do implementacji algorytmów przetwarzania obrazów wielospektralnych.

\index{OpenCV} \section{OpenCV}
OpenCV (Open Source Computer Vision Library) to biblioteka przeznaczona do rozpoznawania obrazów oraz uczenia maszynowego~\cite{opencv}.

Biblioteka ta znajduje wykorzystanie w systemie Gerbil jako bogata baza struktur danych wykorzystywanych do przetwarzania obrazów oraz zaawansowanych algorytmów rozpoznawania obrazów.